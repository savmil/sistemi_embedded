\selectlanguage{italian}%

\section{Soluzione}

La minimizzazione delle funzioni sar� effettuata attraverso i differenti
metodi indicati.

\subsection{Minimizzazione 1}

\label{subsec:min1}

La prima funzione � stata minimizzata attraverso il metodo delle mappe
di Karnaugh.

\begin{table}[H]
Si pu� generare la seguente tabella di verit� a partire dalla definizione
insiemistica:

\begin{minipage}[t]{0.5\columnwidth}%
\begin{longtable}{|>{\centering}p{1cm}||>{\centering}p{1cm}||>{\centering}p{1cm}||>{\centering}p{1cm}||>{\centering}p{1cm}|}
\hline 
X & Y & Z & V & F\tabularnewline
\hline 
\hline 
0 & 0 & 0 & 0 & 1\tabularnewline
\hline 
\hline 
0 & 0 & 0 & 1 & 0\tabularnewline
\hline 
\hline 
0 & 0 & 1 & 0 & 1\tabularnewline
\hline 
\hline 
0 & 0 & 1 & 1 & 0\tabularnewline
\hline 
\hline 
0 & 1 & 0 & 0 & 1\tabularnewline
\hline 
\hline 
0 & 1 & 0 & 1 & 0\tabularnewline
\hline 
\hline 
0 & 1 & 1 & 0 & 0\tabularnewline
\hline 
\hline 
0 & 1 & 1 & 1 & -\tabularnewline
\hline 
\hline 
1 & 0 & 0 & 0 & 1\tabularnewline
\hline 
\hline 
1 & 0 & 0 & 1 & 0\tabularnewline
\hline 
\hline 
1 & 0 & 1 & 0 & 1\tabularnewline
\hline 
\hline 
1 & 0 & 1 & 1 & 1\tabularnewline
\hline 
\hline 
1 & 1 & 0 & 0 & 0\tabularnewline
\hline 
\hline 
1 & 1 & 0 & 1 & 0\tabularnewline
\hline 
\hline 
1 & 1 & 1 & 0 & -\tabularnewline
\hline 
1 & 1 & 1 & 1 & 1\tabularnewline
\hline 
\end{longtable}%
\end{minipage}\hfill{}

\begin{minipage}[t]{0.5\columnwidth}%
\begin{Karnaugh}
       \contingut{1, , 1, , 1, , , -, 1, , 1,1 , , , -,1}
       \implicant{0}{4}{red}% da impl a impl passando davanti
	\implicant{15}{10}{blue}
       \implicantcostats{0}{2}{green}
       \implicantcostats{8}{10}{green}
        % da impl a impl passando per dietro
 \end{Karnaugh}

%
\end{minipage}

\end{table}

Individuando i sottocubi di area massima e gli implicanti primi essenziali,
perveniamo alla funzione minimizzata:

\[
F=\neg x\neg z\neg v+\neg y\neg v+xz
\]

\subsection{Minimizzazione 2}

\label{subsec:min2}

Al fine di minimizzare questa funzione � stato utilizzato il metodo
di Quine Mc Cluskey.

\begin{table}[H]
Si pu� generare la seguente tabella di verit� a partire dalla definizione
insiemistica:

\begin{minipage}[t]{0.5\columnwidth}%
\begin{longtable}{|>{\centering}p{1cm}||>{\centering}p{1cm}||>{\centering}p{1cm}||>{\centering}p{1cm}||>{\centering}p{1cm}|}
\hline 
X & Y & Z & V & F\tabularnewline
\hline 
\hline 
0 & 0 & 0 & 0 & 1\tabularnewline
\hline 
\hline 
0 & 0 & 0 & 1 & 1\tabularnewline
\hline 
\hline 
0 & 0 & 1 & 0 & 1\tabularnewline
\hline 
\hline 
0 & 0 & 1 & 1 & 0\tabularnewline
\hline 
\hline 
0 & 1 & 0 & 0 & 0\tabularnewline
\hline 
\hline 
0 & 1 & 0 & 1 & -\tabularnewline
\hline 
\hline 
0 & 1 & 1 & 0 & 0\tabularnewline
\hline 
\hline 
0 & 1 & 1 & 1 & 1\tabularnewline
\hline 
\hline 
1 & 0 & 0 & 0 & 1\tabularnewline
\hline 
\hline 
1 & 0 & 0 & 1 & -\tabularnewline
\hline 
\hline 
1 & 0 & 1 & 0 & 1\tabularnewline
\hline 
\hline 
1 & 0 & 1 & 1 & -\tabularnewline
\hline 
\hline 
1 & 1 & 0 & 0 & 0\tabularnewline
\hline 
\hline 
1 & 1 & 0 & 1 & 0\tabularnewline
\hline 
\hline 
1 & 1 & 1 & 0 & 0\tabularnewline
\hline 
1 & 1 & 1 & 1 & 1\tabularnewline
\hline 
\end{longtable}%
\end{minipage}
\end{table}

Seguiamo brevemente le fasi di espansione e copertura del metodo di
minimizzazione applicato.

\begin{table}[H]
\noindent\begin{minipage}[t]{1\columnwidth}%
\begin{tabular}{|c|c|c|c|c|c|}
\hline 
implicante & x & y & z & v & check\tabularnewline
\hline 
\hline 
0 & 0 & 0 & 0 & 0 & $\checkmark$\tabularnewline
\hline 
1 & 0 & 0 & 0 & 1 & $\checkmark$\tabularnewline
2 & 0 & 0 & 1 & 0 & $\checkmark$\tabularnewline
8 & 1 & 0 & 0 & 0 & $\checkmark$\tabularnewline
\hline 
5 & 0 & 1 & 0 & 1 & $\checkmark$\tabularnewline
9 & 1 & 0 & 0 & 1 & $\checkmark$\tabularnewline
10 & 1 & 0 & 1 & 0 & $\checkmark$\tabularnewline
\hline 
7 & 0 & 1 & 1 & 1 & $\checkmark$\tabularnewline
11 & 1 & 0 & 1 & 1 & $\checkmark$\tabularnewline
\hline 
15 & 1 & 1 & 1 & 1 & $\checkmark$\tabularnewline
\hline 
\end{tabular}\qquad{}\qquad{}%
\begin{tabular}{|c|c|c|c|c|c|}
\hline 
implicante & x & y & z & v & check\tabularnewline
\hline 
\hline 
0,1 & 0 & 0 & 0 & - & $\checkmark$\tabularnewline
0,2 & 0 & 0 & - & 0 & $\checkmark$\tabularnewline
0,8 & - & 0 & 0 & 0 & $\checkmark$\tabularnewline
\hline 
1,5 & 0 & - & 0 & 1 & a\tabularnewline
1,9 & - & 0 & 0 & 1 & $\checkmark$\tabularnewline
2,10 & - & 0 & 1 & 0 & $\checkmark$\tabularnewline
8,9 & 1 & 0 & 0 & - & $\checkmark$\tabularnewline
8,10 & 1 & 0 & - & 0 & $\checkmark$\tabularnewline
\hline 
5,7 & 0 & 1 & - & 1 & b\tabularnewline
9,11 & 1 & 0 & - & 1 & $\checkmark$\tabularnewline
10,11 & 1 & 0 & 1 & - & $\checkmark$\tabularnewline
\hline 
7,15 & - & 1 & 1 & 1 & c\tabularnewline
11,15 & 1 & - & 1 & 1 & d\tabularnewline
\hline 
\end{tabular}%
\end{minipage}
\end{table}

\begin{table}[H]
\begin{minipage}[t]{0.5\columnwidth}%
\begin{tabular}{|c|c|c|c|c|c|}
\hline 
implicante & x & y & z & v & check\tabularnewline
\hline 
\hline 
0,1,8,9 & - & 0 & 0 & - & e\tabularnewline
0,2,8,10 & - & 0 & - & 0 & f\tabularnewline
\hline 
8,9,10,11 & 1 & 0 & - & - & g\tabularnewline
\hline 
\end{tabular}\medskip{}
\bigskip{}
%
\end{minipage}

Individuati gli implicanti primi al termine della fase di espansione,
si prosegue con la fase di copertura.

\medskip{}

\noindent\begin{minipage}[t]{1\columnwidth}%
\begin{tabular}{|c|c|c|c|c|c|c|c|}
\hline 
 & 0 & 1 & 2 & 7 & 8 & 10 & 15\tabularnewline
\hline 
\hline 
a &  & X &  &  &  &  & \tabularnewline
\hline 
b &  &  &  & X &  &  & \tabularnewline
\hline 
c &  &  &  & X &  &  & X\tabularnewline
\hline 
d &  &  &  &  &  &  & X\tabularnewline
\hline 
e & X & X &  &  & X &  & \tabularnewline
\hline 
f & X &  & X &  & X & X & \tabularnewline
\hline 
g &  &  &  &  & X & X & \tabularnewline
\hline 
\end{tabular}\qquad{}\qquad{}%
\begin{tabular}{|c|c|}
\hline 
 & 1\tabularnewline
\hline 
\hline 
a & X\tabularnewline
\hline 
e & X\tabularnewline
\hline 
\end{tabular}%
\end{minipage}
\end{table}

Discriminando gli implicanti primi essenziali primari ($f$) e secondari
($c$), questi ultimi mediante le regole di dominanza, si ottiene
una tabella cliclica non ulteriormente riducibile. Applicando il metodo
Branch\&Bound in questo caso banale, si prosegue nella scelta dell'implicante
$e$ per il completamento della copertura. Questa scelta si basa sul
costo in termini di letterali dei due implicanti, che individua $e$
come l'implicante che consente la copertura a costo minimo.

\begin{figure}[H]
\centering
\includegraphics[scale=0.7]{esercizio01/images/B&B.png}
\end{figure}

Analogamente, tramite il metodo di Petrick, otteniamo la somma degli
implicanti che coprono il mintermine in questione:

\[
A+E=1
\]

Tra essi si effettua la medesima scelta dettata dal costo dei letterali.

La funzione di uscita, dunque, � pari a:

\[
F=c+e+f=yzv+\neg y\neg z+\neg y\neg v
\]

\subsection{Minimizzazione 3}

\label{subsec:min3}

Si prosegue nuovamente con metodo Quine Mc Cluskey.

\begin{table}[H]
Si pu� generare la seguente tabella di verit� a partire dalla definizione
insiemistica:

\begin{minipage}[t]{0.5\columnwidth}%
\begin{longtable}{|>{\centering}p{1cm}||>{\centering}p{1cm}||>{\centering}p{1cm}||>{\centering}p{1cm}||>{\centering}p{1cm}|}
\hline 
X & Y & Z & V & F\tabularnewline
\hline 
\hline 
0 & 0 & 0 & 0 & 0\tabularnewline
\hline 
\hline 
0 & 0 & 0 & 1 & 1\tabularnewline
\hline 
\hline 
0 & 0 & 1 & 0 & -\tabularnewline
\hline 
\hline 
0 & 0 & 1 & 1 & 1\tabularnewline
\hline 
\hline 
0 & 1 & 0 & 0 & 0\tabularnewline
\hline 
\hline 
0 & 1 & 0 & 1 & 1\tabularnewline
\hline 
\hline 
0 & 1 & 1 & 0 & 0\tabularnewline
\hline 
\hline 
0 & 1 & 1 & 1 & 1\tabularnewline
\hline 
\hline 
1 & 0 & 0 & 0 & 1\tabularnewline
\hline 
\hline 
1 & 0 & 0 & 1 & 1\tabularnewline
\hline 
\hline 
1 & 0 & 1 & 0 & 0\tabularnewline
\hline 
\hline 
1 & 0 & 1 & 1 & 1\tabularnewline
\hline 
\hline 
1 & 1 & 0 & 0 & -\tabularnewline
\hline 
\hline 
1 & 1 & 0 & 1 & 1\tabularnewline
\hline 
\hline 
1 & 1 & 1 & 0 & 1\tabularnewline
\hline 
1 & 1 & 1 & 1 & 1\tabularnewline
\hline 
\end{longtable}%
\end{minipage}
\end{table}

Attraverso la minimizzazione riusciamo ad ottenere un insieme di implicanti
primi essenziali che coprono totalmente la funzione:

\[
F=v+x\lnot z+xy
\]

\begin{comment}
y=$d+h+a+boe$ 

a=$\neg xv+\neg yv+x\neg z+xy$

e=$zv$
\end{comment}

\subsection{Minimizzazione 4}

\label{subsec:min4}

Poich� non si ricade pi� nel caso a singola uscita, si � adottato
il metodo di Quine Mc Cluskey per funzioni a pi� uscite. Esso consente
il riuso di implicanti comuni a pi� funzioni, che non si avrebbe per
semplice minimizzazione individuale.

\begin{table}[H]
Si pu� generare la seguente tabella di verit� a partire dalla definizione
insiemistica:

\begin{minipage}[t]{0.5\columnwidth}%
\begin{longtable}[r]{|>{\centering}p{1cm}||>{\centering}p{1cm}||>{\centering}p{1cm}||>{\centering}p{1cm}||>{\centering}p{1cm}||>{\centering}p{1cm}||>{\centering}p{1cm}|}
\hline 
X & Y & Z & V & F1 & F2 & F3\tabularnewline
\hline 
\hline 
0 & 0 & 0 & 0 & 1 & 1 & 0\tabularnewline
\hline 
\hline 
0 & 0 & 0 & 1 & - & 1 & 0\tabularnewline
\hline 
\hline 
0 & 0 & 1 & 0 & - & 0 & 0\tabularnewline
\hline 
\hline 
0 & 0 & 1 & 1 & 0 & 0 & 0\tabularnewline
\hline 
\hline 
0 & 1 & 0 & 0 & 1 & - & 0\tabularnewline
\hline 
\hline 
0 & 1 & 0 & 1 & - & 0 & -\tabularnewline
\hline 
\hline 
0 & 1 & 1 & 0 & 0 & 0 & 0\tabularnewline
\hline 
\hline 
0 & 1 & 1 & 1 & - & 1 & 1\tabularnewline
\hline 
\hline 
1 & 0 & 0 & 0 & 1 & 0 & 0\tabularnewline
\hline 
\hline 
1 & 0 & 0 & 1 & 1 & 1 & 1\tabularnewline
\hline 
\hline 
1 & 0 & 1 & 0 & 0 & 0 & 1\tabularnewline
\hline 
\hline 
1 & 0 & 1 & 1 & 0 & 1 & 1\tabularnewline
\hline 
\hline 
1 & 1 & 0 & 0 & 0 & - & -\tabularnewline
\hline 
\hline 
1 & 1 & 0 & 1 & 0 & 1 & 0\tabularnewline
\hline 
\hline 
1 & 1 & 1 & 0 & 1 & 0 & 0\tabularnewline
\hline 
\hline 
1 & 1 & 1 & 1 & 1 & 0 & 1\tabularnewline
\hline 
\end{longtable}%
\end{minipage}
\end{table}

Percorriamo l'applicazione del metodo che, rispetto al caso a singola
uscita, presenta l'introduzione della maschera rappresentativa delle
funzioni a cui l'implicante fa riferimento.

\begin{table}[H]
\begin{tabular}{|c||c|c|c|c||c|c|c||c|}
\hline 
implicante & x & y & z & v & F1 & F2 & F3 & check\tabularnewline
\hline 
0 & 0 & 0 & 0 & 0 & 1 & 1 & 0 & $\checkmark$\tabularnewline
\hline 
\hline 
1 & 0 & 0 & 0 & 1 & - & 1 & 0 & $\checkmark$\tabularnewline
\hline 
2 & 0 & 0 & 1 & 0 & - & 0 & 0 & $\checkmark$\tabularnewline
\hline 
4 & 0 & 1 & 0 & 0 & 1 & - & 0 & $\checkmark$\tabularnewline
\hline 
8 & 1 & 0 & 0 & 0 & 1 & 0 & 0 & $\checkmark$\tabularnewline
\hline 
\hline 
5 & 0 & 1 & 0 & 1 & - & 0 & - & $\checkmark$\tabularnewline
\hline 
9 & 1 & 0 & 0 & 1 & 1 & 1 & 1 & $a$\tabularnewline
\hline 
10 & 1 & 0 & 1 & 0 & 0 & 0 & 1 & $\checkmark$\tabularnewline
\hline 
12 & 1 & 1 & 0 & 0 & 0 & - & - & $\checkmark$\tabularnewline
\hline 
\hline 
7 & 0 & 1 & 1 & 1 & - & 1 & 1 & $b$\tabularnewline
\hline 
11 & 1 & 0 & 1 & 1 & 0 & 1 & 1 & $\checkmark$\tabularnewline
\hline 
13 & 1 & 1 & 0 & 1 & 0 & 1 & 0 & $\checkmark$\tabularnewline
\hline 
14 & 1 & 1 & 1 & 0 & 1 & 0 & 0 & $\checkmark$\tabularnewline
\hline 
\hline 
15 & 1 & 1 & 1 & 1 & 1 & 0 & 1 & $\checkmark$\tabularnewline
\hline 
\end{tabular}\bigskip{}
\\
\bigskip{}
\begin{tabular}{|c||c|c|c|c||c|c|c||c|}
\hline 
implicante & x & y & z & v & F1 & F2 & F3 & check\tabularnewline
\hline 
0, 1 & 0 & 0 & 0 & - & 1 & 1 & 0 & $c$\tabularnewline
\hline 
0, 2 & 0 & 0 & - & 0 & 1 & 0 & 0 & $d$\tabularnewline
\hline 
0, 4 & 0 & - & 0 & 0 & 1 & 1 & 0 & $e$\tabularnewline
\hline 
0, 8 & - & 0 & 0 & 0 & 1 & 0 & 0 & $\checkmark$\tabularnewline
\hline 
\hline 
1, 5 & 0 & - & 0 & 1 & - & 0 & 0 & $\checkmark$\tabularnewline
\hline 
1, 9 & - & 0 & 0 & 1 & 1 & 1 & 0 & $f$\tabularnewline
\hline 
4, 5 & 0 & 1 & 0 & - & 1 & 0 & 0 & $\checkmark$\tabularnewline
\hline 
4, 12 & - & 1 & 0 & 0 & 0 & - & 0 & $\checkmark$\tabularnewline
\hline 
8, 9 & 1 & 0 & 0 & - & 1 & 0 & 0 & $\checkmark$\tabularnewline
\hline 
\hline 
5, 7 & 0 & 1 & - & 1 & - & 0 & 1 & $g$\tabularnewline
\hline 
9, 11 & 1 & 0 & - & 1 & 0 & 1 & 1 & $h$\tabularnewline
\hline 
9, 13 & 1 & - & 0 & 1 & 0 & 1 & 0 & $i$\tabularnewline
\hline 
10, 11 & 1 & 0 & 1 & - & 0 & 0 & 1 & $l$\tabularnewline
\hline 
12, 13 & 1 & 1 & 0 & - & 0 & 1 & 0 & $m$\tabularnewline
\hline 
\hline 
7, 15 & - & 1 & 1 & 1 & 1 & 0 & 1 & $n$\tabularnewline
\hline 
11, 15 & 1 & - & 1 & 1 & 0 & 0 & 1 & $o$\tabularnewline
\hline 
14, 15 & 1 & 1 & 1 & - & 1 & 0 & 0 & $p$\tabularnewline
\hline 
\end{tabular}\\
\bigskip{}
\begin{tabular}{|c||c|c|c|c||c|c|c||c|}
\hline 
implicante & x & y & z & v & F1 & F2 & F3 & check\tabularnewline
\hline 
0, 1, 4, 5 & 0 & - & 0 & - & 1 & 0 & 0 & $q$\tabularnewline
\hline 
0, 1, 8, 9 & - & 0 & 0 & - & 1 & 0 & 0 & $r$\tabularnewline
\hline 
\end{tabular}
\end{table}

Segue la fase di copertura con gli implicanti primi trovati. Essa
viene effettuata con l'obiettivo di minimizzare il costo dei letterali.

\begin{table}[H]
\begin{tabular}{|c||c|c|c|c|c|c||c|c|c|c|c|c||c|c|c|c|c||c|}
\hline 
 & \multicolumn{6}{c||}{F1} & \multicolumn{6}{c||}{F2} & \multicolumn{5}{c||}{F3} & costo\tabularnewline
\hline 
\hline 
 & 0 & 4 & 8 & 9 & 14 & 15 & 0 & 1 & 7 & 9 & 11 & 13 & 7 & 9 & 10 & 11 & 15 & \tabularnewline
\hline 
a &  &  &  & X &  &  &  &  &  & X &  &  &  & X &  &  &  & 4\tabularnewline
\hline 
b &  &  &  &  &  &  &  &  & X &  &  &  & X &  &  &  &  & 4\tabularnewline
\hline 
c & X &  &  &  &  &  & X & X &  &  &  &  &  &  &  &  &  & 3\tabularnewline
\hline 
d & X &  &  &  &  &  &  &  &  &  &  &  &  &  &  &  &  & 3\tabularnewline
\hline 
e & X & X &  &  &  &  & X &  &  &  &  &  &  &  &  &  &  & 3\tabularnewline
\hline 
f &  &  &  & X &  &  &  & X &  & X &  &  &  &  &  &  &  & 3\tabularnewline
\hline 
g &  &  &  &  &  &  &  &  &  &  &  &  & X &  &  &  &  & 3\tabularnewline
\hline 
h &  &  &  &  &  &  &  &  &  & X & X &  &  & X &  & X &  & 3\tabularnewline
\hline 
i &  &  &  &  &  &  &  &  &  & X &  & X &  &  &  &  &  & 3\tabularnewline
\hline 
l &  &  &  &  &  &  &  &  &  &  &  &  &  &  & X & X &  & 3\tabularnewline
\hline 
m &  &  &  &  &  &  &  &  &  &  &  & X &  &  &  &  &  & 3\tabularnewline
\hline 
n &  &  &  &  &  & X &  &  &  &  &  &  & X &  &  &  & X & 3\tabularnewline
\hline 
o &  &  &  &  &  &  &  &  &  &  &  &  &  &  &  & X & X & 3\tabularnewline
\hline 
p &  &  &  &  & X & X &  &  &  &  &  &  &  &  &  &  &  & 3\tabularnewline
\hline 
q & X & X &  &  &  &  &  &  &  &  &  &  &  &  &  &  &  & 2\tabularnewline
\hline 
r & X &  & X & X &  &  &  &  &  &  &  &  &  &  &  &  &  & 2\tabularnewline
\hline 
\end{tabular}

\bigskip{}

\bigskip{}

\begin{tabular}{|c||c||c|c|c||c|c|c||c|}
\hline 
 & \multicolumn{1}{c||}{F1} & \multicolumn{3}{c||}{F2} & \multicolumn{3}{c||}{F3} & costo\tabularnewline
\hline 
\hline 
 & 4 & 0 & 1 & 13 & 7 & 9 & 15 & \tabularnewline
\hline 
a &  &  &  &  &  & X &  & 4\tabularnewline
\hline 
b &  &  &  &  & X &  &  & 1\tabularnewline
\hline 
c &  & X & X &  &  &  &  & 3\tabularnewline
\hline 
e & X & X &  &  &  &  &  & 3\tabularnewline
\hline 
f &  &  & X &  &  &  &  & 3\tabularnewline
\hline 
g &  &  &  &  & X &  &  & 3\tabularnewline
\hline 
h &  &  &  &  &  & X &  & 1\tabularnewline
\hline 
i &  &  &  & X &  &  &  & 3\tabularnewline
\hline 
m &  &  &  & X &  &  &  & 3\tabularnewline
\hline 
n &  &  &  &  & X &  & X & 3\tabularnewline
\hline 
o &  &  &  &  &  &  & X & 3\tabularnewline
\hline 
q & X &  &  &  &  &  &  & 2\tabularnewline
\hline 
\end{tabular}\qquad{}\qquad{}%
\begin{tabular}{|c||c||c|c|c||c|c|c||c|}
\hline 
 & \multicolumn{1}{c||}{F1} & \multicolumn{3}{c||}{F2} & \multicolumn{3}{c||}{F3} & costo\tabularnewline
\hline 
\hline 
 & 4 & 0 & 1 & 13 & 7 & 9 & 15 & \tabularnewline
\hline 
b &  &  &  &  & X &  &  & 1\tabularnewline
\hline 
c &  & X & X &  &  &  &  & 3\tabularnewline
\hline 
e & X & X &  &  &  &  &  & 3\tabularnewline
\hline 
h &  &  &  &  &  & X &  & 1\tabularnewline
\hline 
i &  &  &  & X &  &  &  & 3\tabularnewline
\hline 
m &  &  &  & X &  &  &  & 3\tabularnewline
\hline 
n &  &  &  &  & X &  & X & 3\tabularnewline
\hline 
q & X &  &  &  &  &  &  & 2\tabularnewline
\hline 
\end{tabular}

\bigskip{}
\bigskip{}

\begin{tabular}{|c||c||c||c|}
\hline 
 & \multicolumn{1}{c||}{F1} & \multicolumn{1}{c||}{F2} & costo\tabularnewline
\hline 
\hline 
 & 4 & 13 & \tabularnewline
\hline 
e & X &  & 3\tabularnewline
\hline 
i &  & X & 3\tabularnewline
\hline 
m &  & X & 3\tabularnewline
\hline 
q & X &  & 2\tabularnewline
\hline 
\end{tabular}\qquad{}\qquad{}%
\begin{tabular}{|c||c||c||c|}
\hline 
 & \multicolumn{1}{c||}{F1} & \multicolumn{1}{c||}{F2} & costo\tabularnewline
\hline 
\hline 
 & 4 & 13 & \tabularnewline
\hline 
i &  & X & 3\tabularnewline
\hline 
m &  & X & 3\tabularnewline
\hline 
q & X &  & 2\tabularnewline
\hline 
\end{tabular}\qquad{}\qquad{}%
\begin{tabular}{|c||c||c|}
\hline 
 & \multicolumn{1}{c||}{F2} & costo\tabularnewline
\hline 
\hline 
 & 13 & \tabularnewline
\hline 
i & X & 3\tabularnewline
\hline 
m & X & 3\tabularnewline
\hline 
\end{tabular}
\end{table}

Si arriva ad una tabella ciclica, i cui implicanti presentano anche
lo stesso costo. Per la minimizzazione si � scelto $i$, ottenendo:

\[
F1=r+p+q=\lnot y\lnot z+xyz+\lnot x\lnot z
\]

\[
F2=b+h+c+i=\lnot xyzv+x\lnot yv+\lnot x\lnot y\lnot z+x\lnot zv
\]

\[
F3=l+n+h=x\lnot yz+yzv+x\lnot yv
\]
\selectlanguage{italian}%

