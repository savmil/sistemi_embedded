\hypertarget{UART__interrupt__uio_8h}{}\section{/media/saverio/\+O\+S/\+Users/\+Saverio/\+Desktop/\+S\+E/git/codici\+\_\+da\+\_\+mandare/\+F\+P\+G\+A/\+U\+A\+R\+T/\+Driver/\+U\+I\+O/\+U\+A\+R\+T\+\_\+interrupt\+\_\+uio.h File Reference}
\label{UART__interrupt__uio_8h}\index{/media/saverio/\+O\+S/\+Users/\+Saverio/\+Desktop/\+S\+E/git/codici\+\_\+da\+\_\+mandare/\+F\+P\+G\+A/\+U\+A\+R\+T/\+Driver/\+U\+I\+O/\+U\+A\+R\+T\+\_\+interrupt\+\_\+uio.\+h@{/media/saverio/\+O\+S/\+Users/\+Saverio/\+Desktop/\+S\+E/git/codici\+\_\+da\+\_\+mandare/\+F\+P\+G\+A/\+U\+A\+R\+T/\+Driver/\+U\+I\+O/\+U\+A\+R\+T\+\_\+interrupt\+\_\+uio.\+h}}


header file U\+A\+R\+T\+\_\+interrupt\+\_\+uio  




\subsection{Detailed Description}
header file U\+A\+R\+T\+\_\+interrupt\+\_\+uio 



\subsection{Function Documentation}
\mbox{\Hypertarget{UART__interrupt__uio_8h_a5fa96fe10ac62639ffae7412b23fd1fa}\label{UART__interrupt__uio_8h_a5fa96fe10ac62639ffae7412b23fd1fa}} 
\index{U\+A\+R\+T\+\_\+interrupt\+\_\+uio.\+h@{U\+A\+R\+T\+\_\+interrupt\+\_\+uio.\+h}!read\+\_\+reg@{read\+\_\+reg}}
\index{read\+\_\+reg@{read\+\_\+reg}!U\+A\+R\+T\+\_\+interrupt\+\_\+uio.\+h@{U\+A\+R\+T\+\_\+interrupt\+\_\+uio.\+h}}
\subsubsection{\texorpdfstring{read\+\_\+reg()}{read\_reg()}}
{\footnotesize\ttfamily unsigned int read\+\_\+reg (\begin{DoxyParamCaption}\item[{void $\ast$}]{addr,  }\item[{unsigned int}]{offset }\end{DoxyParamCaption})}



Utilizzata per leggere un valore da un registro della periferica, specificando l\textquotesingle{}indirizzo base virtuale e l\textquotesingle{}offset del registro da cui leggere. 


\begin{DoxyParams}{Parameters}
{\em addr,puntatore} & all\textquotesingle{} indirizzo da voler leggere \\
\hline
{\em offset,offset} & a partire dall\textquotesingle{} indirizzo a cui vogliamo scrivere\\
\hline
{\em addr} & indirizzo virtuale della periferica \\
\hline
{\em offset} & offset del registro a cui leggere\\
\hline
\end{DoxyParams}
\begin{DoxyReturn}{Returns}
valore presente all\textquotesingle{}interno del registro 
\end{DoxyReturn}
\mbox{\Hypertarget{UART__interrupt__uio_8h_aa8b2118cad276d61a22abef215618c49}\label{UART__interrupt__uio_8h_aa8b2118cad276d61a22abef215618c49}} 
\index{U\+A\+R\+T\+\_\+interrupt\+\_\+uio.\+h@{U\+A\+R\+T\+\_\+interrupt\+\_\+uio.\+h}!wait\+\_\+for\+\_\+interrupt@{wait\+\_\+for\+\_\+interrupt}}
\index{wait\+\_\+for\+\_\+interrupt@{wait\+\_\+for\+\_\+interrupt}!U\+A\+R\+T\+\_\+interrupt\+\_\+uio.\+h@{U\+A\+R\+T\+\_\+interrupt\+\_\+uio.\+h}}
\subsubsection{\texorpdfstring{wait\+\_\+for\+\_\+interrupt()}{wait\_for\_interrupt()}}
{\footnotesize\ttfamily void wait\+\_\+for\+\_\+interrupt (\begin{DoxyParamCaption}\item[{struct pollfd $\ast$}]{poll\+\_\+fds,  }\item[{void $\ast$}]{uart\+\_\+rx\+\_\+ptr,  }\item[{void $\ast$}]{uart\+\_\+tx\+\_\+ptr }\end{DoxyParamCaption})}



Attende l\textquotesingle{} arrivo di un interrupt utilizzando la read su un device U\+IO. 


\begin{DoxyParams}{Parameters}
{\em poll\+\_\+fds} & struct contenente i due descrittori del file per i due device \hyperlink{structUART}{U\+A\+RT} \\
\hline
{\em uart\+\_\+rx\+\_\+ptr} & indirizzo virtuale della periferica \hyperlink{structUART}{U\+A\+RT} utilizzata in ricezione \\
\hline
{\em uart\+\_\+tx\+\_\+ptr} & indirizzo virtuale della periferica \hyperlink{structUART}{U\+A\+RT} utilizzata in trasmissione \\
\hline
\end{DoxyParams}
Se vi è un\textquotesingle{}interruzione sul device U\+I\+O0 associato all\textquotesingle{}\hyperlink{structUART}{U\+A\+RT} per la ricezione

Se vi è un\textquotesingle{}interruzione sul device U\+I\+O0 associato all\textquotesingle{}\hyperlink{structUART}{U\+A\+RT} per la trasmissione \mbox{\Hypertarget{UART__interrupt__uio_8h_a19948bb35adde231ba424f7bc14289ee}\label{UART__interrupt__uio_8h_a19948bb35adde231ba424f7bc14289ee}} 
\index{U\+A\+R\+T\+\_\+interrupt\+\_\+uio.\+h@{U\+A\+R\+T\+\_\+interrupt\+\_\+uio.\+h}!write\+\_\+reg@{write\+\_\+reg}}
\index{write\+\_\+reg@{write\+\_\+reg}!U\+A\+R\+T\+\_\+interrupt\+\_\+uio.\+h@{U\+A\+R\+T\+\_\+interrupt\+\_\+uio.\+h}}
\subsubsection{\texorpdfstring{write\+\_\+reg()}{write\_reg()}}
{\footnotesize\ttfamily void write\+\_\+reg (\begin{DoxyParamCaption}\item[{void $\ast$}]{addr,  }\item[{unsigned int}]{offset,  }\item[{unsigned int}]{value }\end{DoxyParamCaption})}



Utilizzata per scrivere un valore all\textquotesingle{}interno di un registro della periferica, specificando l\textquotesingle{}indirizzo base virtuale e l\textquotesingle{}offset del registro in cui scrivere. 


\begin{DoxyParams}{Parameters}
{\em addr,puntatore} & all\textquotesingle{} indirizzo da voler scrivere \\
\hline
{\em offset,offset} & a partire dall\textquotesingle{} indirizzo a cui vogliamo scrivere \\
\hline
{\em value,valore} & da voler scrivere\\
\hline
{\em addr} & indirizzo virtuale della periferica \\
\hline
{\em offset} & offset del registro a cui scrivere \\
\hline
{\em valore} & da scrivere \\
\hline
\end{DoxyParams}
