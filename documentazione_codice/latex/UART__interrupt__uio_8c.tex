\hypertarget{UART__interrupt__uio_8c}{}\section{/media/saverio/\+O\+S/\+Users/\+Saverio/\+Desktop/\+S\+E/git/codici\+\_\+da\+\_\+mandare/\+F\+P\+G\+A/\+U\+A\+R\+T/\+Driver/\+U\+I\+O/\+U\+A\+R\+T\+\_\+interrupt\+\_\+uio.c File Reference}
\label{UART__interrupt__uio_8c}\index{/media/saverio/\+O\+S/\+Users/\+Saverio/\+Desktop/\+S\+E/git/codici\+\_\+da\+\_\+mandare/\+F\+P\+G\+A/\+U\+A\+R\+T/\+Driver/\+U\+I\+O/\+U\+A\+R\+T\+\_\+interrupt\+\_\+uio.\+c@{/media/saverio/\+O\+S/\+Users/\+Saverio/\+Desktop/\+S\+E/git/codici\+\_\+da\+\_\+mandare/\+F\+P\+G\+A/\+U\+A\+R\+T/\+Driver/\+U\+I\+O/\+U\+A\+R\+T\+\_\+interrupt\+\_\+uio.\+c}}


permette la gestione della periferica \hyperlink{structUART}{U\+A\+RT} utilizzando un driver di tipo U\+IO  




\subsection{Detailed Description}
permette la gestione della periferica \hyperlink{structUART}{U\+A\+RT} utilizzando un driver di tipo U\+IO 



\subsection{Function Documentation}
\mbox{\Hypertarget{UART__interrupt__uio_8c_ab9734b87bdd5937a4161587344636fd2}\label{UART__interrupt__uio_8c_ab9734b87bdd5937a4161587344636fd2}} 
\index{U\+A\+R\+T\+\_\+interrupt\+\_\+uio.\+c@{U\+A\+R\+T\+\_\+interrupt\+\_\+uio.\+c}!read\+\_\+reg@{read\+\_\+reg}}
\index{read\+\_\+reg@{read\+\_\+reg}!U\+A\+R\+T\+\_\+interrupt\+\_\+uio.\+c@{U\+A\+R\+T\+\_\+interrupt\+\_\+uio.\+c}}
\subsubsection{\texorpdfstring{read\+\_\+reg()}{read\_reg()}}
{\footnotesize\ttfamily unsigned int read\+\_\+reg (\begin{DoxyParamCaption}\item[{void $\ast$}]{addr,  }\item[{unsigned int}]{offset }\end{DoxyParamCaption})}



Utilizzata per leggere un valore da un registro della periferica, specificando l\textquotesingle{}indirizzo base virtuale e l\textquotesingle{}offset del registro da cui leggere. 


\begin{DoxyParams}{Parameters}
{\em addr} & indirizzo virtuale della periferica \\
\hline
{\em offset} & offset del registro a cui leggere\\
\hline
\end{DoxyParams}
\begin{DoxyReturn}{Returns}
valore presente all\textquotesingle{}interno del registro 
\end{DoxyReturn}
\mbox{\Hypertarget{UART__interrupt__uio_8c_ae652d21ba3de22e6343ae7bde0d608a7}\label{UART__interrupt__uio_8c_ae652d21ba3de22e6343ae7bde0d608a7}} 
\index{U\+A\+R\+T\+\_\+interrupt\+\_\+uio.\+c@{U\+A\+R\+T\+\_\+interrupt\+\_\+uio.\+c}!wait\+\_\+for\+\_\+interrupt@{wait\+\_\+for\+\_\+interrupt}}
\index{wait\+\_\+for\+\_\+interrupt@{wait\+\_\+for\+\_\+interrupt}!U\+A\+R\+T\+\_\+interrupt\+\_\+uio.\+c@{U\+A\+R\+T\+\_\+interrupt\+\_\+uio.\+c}}
\subsubsection{\texorpdfstring{wait\+\_\+for\+\_\+interrupt()}{wait\_for\_interrupt()}}
{\footnotesize\ttfamily void wait\+\_\+for\+\_\+interrupt (\begin{DoxyParamCaption}\item[{struct pollfd $\ast$}]{poll\+\_\+fds,  }\item[{void $\ast$}]{uart\+\_\+rx\+\_\+ptr,  }\item[{void $\ast$}]{uart\+\_\+tx\+\_\+ptr }\end{DoxyParamCaption})}



Attende l\textquotesingle{} arrivo di un interrupt utilizzando la read su un device U\+IO. 


\begin{DoxyParams}{Parameters}
{\em poll\+\_\+fds} & struct contenente i due descrittori del file per i due device \hyperlink{structUART}{U\+A\+RT} \\
\hline
{\em uart\+\_\+rx\+\_\+ptr} & indirizzo virtuale della periferica \hyperlink{structUART}{U\+A\+RT} utilizzata in ricezione \\
\hline
{\em uart\+\_\+tx\+\_\+ptr} & indirizzo virtuale della periferica \hyperlink{structUART}{U\+A\+RT} utilizzata in trasmissione \\
\hline
\end{DoxyParams}
Se vi è un\textquotesingle{}interruzione sul device U\+I\+O0 associato all\textquotesingle{}\hyperlink{structUART}{U\+A\+RT} per la ricezione

Se vi è un\textquotesingle{}interruzione sul device U\+I\+O0 associato all\textquotesingle{}\hyperlink{structUART}{U\+A\+RT} per la trasmissione \mbox{\Hypertarget{UART__interrupt__uio_8c_a967a78bccad98fd07accc16dd33243fe}\label{UART__interrupt__uio_8c_a967a78bccad98fd07accc16dd33243fe}} 
\index{U\+A\+R\+T\+\_\+interrupt\+\_\+uio.\+c@{U\+A\+R\+T\+\_\+interrupt\+\_\+uio.\+c}!write\+\_\+reg@{write\+\_\+reg}}
\index{write\+\_\+reg@{write\+\_\+reg}!U\+A\+R\+T\+\_\+interrupt\+\_\+uio.\+c@{U\+A\+R\+T\+\_\+interrupt\+\_\+uio.\+c}}
\subsubsection{\texorpdfstring{write\+\_\+reg()}{write\_reg()}}
{\footnotesize\ttfamily void write\+\_\+reg (\begin{DoxyParamCaption}\item[{void $\ast$}]{addr,  }\item[{unsigned int}]{offset,  }\item[{unsigned int}]{value }\end{DoxyParamCaption})}



Utilizzata per scrivere un valore all\textquotesingle{}interno di un registro della periferica, specificando l\textquotesingle{}indirizzo base virtuale e l\textquotesingle{}offset del registro in cui scrivere. 


\begin{DoxyParams}{Parameters}
{\em addr} & indirizzo virtuale della periferica \\
\hline
{\em offset} & offset del registro a cui scrivere \\
\hline
{\em valore} & da scrivere \\
\hline
\end{DoxyParams}
