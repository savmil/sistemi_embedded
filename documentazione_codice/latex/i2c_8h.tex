\hypertarget{i2c_8h}{}\section{/media/saverio/\+O\+S/\+Users/\+Saverio/\+Desktop/\+S\+E/git/codici\+\_\+da\+\_\+mandare/\+S\+T\+M/\+C\+R\+C\+\_\+\+Multi\+Serial/\+Inc/i2c.h File Reference}
\label{i2c_8h}\index{/media/saverio/\+O\+S/\+Users/\+Saverio/\+Desktop/\+S\+E/git/codici\+\_\+da\+\_\+mandare/\+S\+T\+M/\+C\+R\+C\+\_\+\+Multi\+Serial/\+Inc/i2c.\+h@{/media/saverio/\+O\+S/\+Users/\+Saverio/\+Desktop/\+S\+E/git/codici\+\_\+da\+\_\+mandare/\+S\+T\+M/\+C\+R\+C\+\_\+\+Multi\+Serial/\+Inc/i2c.\+h}}


header file per la configurazione della periferica I2C  




\subsection{Detailed Description}
header file per la configurazione della periferica I2C 



\subsection{Function Documentation}
\mbox{\Hypertarget{i2c_8h_abf4a9d7083f94ed76d3042acbc35f40a}\label{i2c_8h_abf4a9d7083f94ed76d3042acbc35f40a}} 
\index{i2c.\+h@{i2c.\+h}!M\+X\+\_\+\+I2\+C2\+\_\+\+Init@{M\+X\+\_\+\+I2\+C2\+\_\+\+Init}}
\index{M\+X\+\_\+\+I2\+C2\+\_\+\+Init@{M\+X\+\_\+\+I2\+C2\+\_\+\+Init}!i2c.\+h@{i2c.\+h}}
\subsubsection{\texorpdfstring{M\+X\+\_\+\+I2\+C2\+\_\+\+Init()}{MX\_I2C2\_Init()}}
{\footnotesize\ttfamily void M\+X\+\_\+\+I2\+C2\+\_\+\+Init (\begin{DoxyParamCaption}\item[{uint16\+\_\+t}]{node\+Address,  }\item[{uint16\+\_\+t}]{group\+Address }\end{DoxyParamCaption})}



Funzione di configurazione della periferica I2C. 


\begin{DoxyParams}{Parameters}
{\em node\+Address} & setta l\textquotesingle{} indentificativo del nodo \\
\hline
{\em group\+Address} & setta l\textquotesingle{} identificato del gruppo a cui il nodo appartiene \\
\hline
\end{DoxyParams}
da ack confrontando tutti i 7 bit dell\textquotesingle{}addres ricevuto con quelli di own\+Address2. utilizzato per realizzare multicast

abilita generic call address. Permette di realizzare broadcast su address 0x00 