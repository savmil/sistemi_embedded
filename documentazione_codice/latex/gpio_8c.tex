\hypertarget{GPIO_8c}{}\section{/media/saverio/\+O\+S/\+Users/\+Saverio/\+Desktop/\+S\+E/git/codici\+\_\+da\+\_\+mandare/\+F\+P\+G\+A/\+G\+P\+I\+O/\+Driver/\+K\+E\+R\+N\+E\+L\+\_\+\+M\+O\+D\+E/\+G\+P\+IO.c File Reference}
\label{GPIO_8c}\index{/media/saverio/\+O\+S/\+Users/\+Saverio/\+Desktop/\+S\+E/git/codici\+\_\+da\+\_\+mandare/\+F\+P\+G\+A/\+G\+P\+I\+O/\+Driver/\+K\+E\+R\+N\+E\+L\+\_\+\+M\+O\+D\+E/\+G\+P\+I\+O.\+c@{/media/saverio/\+O\+S/\+Users/\+Saverio/\+Desktop/\+S\+E/git/codici\+\_\+da\+\_\+mandare/\+F\+P\+G\+A/\+G\+P\+I\+O/\+Driver/\+K\+E\+R\+N\+E\+L\+\_\+\+M\+O\+D\+E/\+G\+P\+I\+O.\+c}}


permette la gestione del singolo \hyperlink{structGPIO}{G\+P\+IO}  




\subsection{Detailed Description}
permette la gestione del singolo \hyperlink{structGPIO}{G\+P\+IO} 



\subsection{Function Documentation}
\mbox{\Hypertarget{GPIO_8c_a35e453e72ddf756e45ee2d312eb760ab}\label{GPIO_8c_a35e453e72ddf756e45ee2d312eb760ab}} 
\index{G\+P\+I\+O.\+c@{G\+P\+I\+O.\+c}!G\+P\+I\+O\+\_\+\+Destroy@{G\+P\+I\+O\+\_\+\+Destroy}}
\index{G\+P\+I\+O\+\_\+\+Destroy@{G\+P\+I\+O\+\_\+\+Destroy}!G\+P\+I\+O.\+c@{G\+P\+I\+O.\+c}}
\subsubsection{\texorpdfstring{G\+P\+I\+O\+\_\+\+Destroy()}{GPIO\_Destroy()}}
{\footnotesize\ttfamily void G\+P\+I\+O\+\_\+\+Destroy (\begin{DoxyParamCaption}\item[{\hyperlink{structGPIO}{G\+P\+IO}$\ast$}]{device }\end{DoxyParamCaption})}



Rimuove un device \hyperlink{structGPIO}{G\+P\+IO} con le relative strutture kernel allocate per il suo funzionamento. 


\begin{DoxyParams}{Parameters}
{\em device} & puntatore a struttura \hyperlink{structGPIO}{G\+P\+IO} che indica l\textquotesingle{}istanza \hyperlink{structGPIO}{G\+P\+IO} da rimuovere \\
\hline
\end{DoxyParams}
\mbox{\Hypertarget{GPIO_8c_a88017c9a3bf24e68865541255a3cf065}\label{GPIO_8c_a88017c9a3bf24e68865541255a3cf065}} 
\index{G\+P\+I\+O.\+c@{G\+P\+I\+O.\+c}!G\+P\+I\+O\+\_\+\+Get\+Device\+Address@{G\+P\+I\+O\+\_\+\+Get\+Device\+Address}}
\index{G\+P\+I\+O\+\_\+\+Get\+Device\+Address@{G\+P\+I\+O\+\_\+\+Get\+Device\+Address}!G\+P\+I\+O.\+c@{G\+P\+I\+O.\+c}}
\subsubsection{\texorpdfstring{G\+P\+I\+O\+\_\+\+Get\+Device\+Address()}{GPIO\_GetDeviceAddress()}}
{\footnotesize\ttfamily void$\ast$ G\+P\+I\+O\+\_\+\+Get\+Device\+Address (\begin{DoxyParamCaption}\item[{\hyperlink{structGPIO}{G\+P\+IO}$\ast$}]{device }\end{DoxyParamCaption})}



Restituisce l\textquotesingle{}indirizzo virtuale di memoria cui è mappato un device. 


\begin{DoxyParams}{Parameters}
{\em device} & puntatore a struttura \hyperlink{structGPIO}{G\+P\+IO}, che si riferisce al device su cui operare \\
\hline
\end{DoxyParams}
\mbox{\Hypertarget{GPIO_8c_abfd5d2d0153f5cbea672c14fb78f500b}\label{GPIO_8c_abfd5d2d0153f5cbea672c14fb78f500b}} 
\index{G\+P\+I\+O.\+c@{G\+P\+I\+O.\+c}!G\+P\+I\+O\+\_\+\+Get\+Poll\+Mask@{G\+P\+I\+O\+\_\+\+Get\+Poll\+Mask}}
\index{G\+P\+I\+O\+\_\+\+Get\+Poll\+Mask@{G\+P\+I\+O\+\_\+\+Get\+Poll\+Mask}!G\+P\+I\+O.\+c@{G\+P\+I\+O.\+c}}
\subsubsection{\texorpdfstring{G\+P\+I\+O\+\_\+\+Get\+Poll\+Mask()}{GPIO\_GetPollMask()}}
{\footnotesize\ttfamily unsigned G\+P\+I\+O\+\_\+\+Get\+Poll\+Mask (\begin{DoxyParamCaption}\item[{\hyperlink{structGPIO}{G\+P\+IO} $\ast$}]{device,  }\item[{struct file $\ast$}]{file\+\_\+ptr,  }\item[{struct poll\+\_\+table\+\_\+struct $\ast$}]{wait }\end{DoxyParamCaption})}



Verifica che le operazioni di lettura risultino non-\/bloccanti. 


\begin{DoxyParams}{Parameters}
{\em device} & puntatore a struttura \hyperlink{structGPIO}{G\+P\+IO}, che si riferisce al device su cui operare \\
\hline
{\em file} & puntatore al descrittore file del device \\
\hline
{\em wait} & puntatore alla struttura poll\+\_\+table\\
\hline
\end{DoxyParams}
\begin{DoxyReturn}{Returns}
maschera di bit che indica se sia possibile effettuare operazioni di lettura non bloccanti.
\end{DoxyReturn}
Back-\/end di tre diverse sys-\/calls\+: poll, epoll e select, \mbox{\Hypertarget{GPIO_8c_a9acce2a93067430307df45f0cc8aa02a}\label{GPIO_8c_a9acce2a93067430307df45f0cc8aa02a}} 
\index{G\+P\+I\+O.\+c@{G\+P\+I\+O.\+c}!G\+P\+I\+O\+\_\+\+Global\+Interrupt\+Disable@{G\+P\+I\+O\+\_\+\+Global\+Interrupt\+Disable}}
\index{G\+P\+I\+O\+\_\+\+Global\+Interrupt\+Disable@{G\+P\+I\+O\+\_\+\+Global\+Interrupt\+Disable}!G\+P\+I\+O.\+c@{G\+P\+I\+O.\+c}}
\subsubsection{\texorpdfstring{G\+P\+I\+O\+\_\+\+Global\+Interrupt\+Disable()}{GPIO\_GlobalInterruptDisable()}}
{\footnotesize\ttfamily void G\+P\+I\+O\+\_\+\+Global\+Interrupt\+Disable (\begin{DoxyParamCaption}\item[{\hyperlink{structGPIO}{G\+P\+IO}$\ast$}]{device }\end{DoxyParamCaption})}



Disabilitazione interrupt globali;. 


\begin{DoxyParams}{Parameters}
{\em device} & puntatore a struttura \hyperlink{structGPIO}{G\+P\+IO}, che si riferisce al device su cui operare \\
\hline
\end{DoxyParams}
\mbox{\Hypertarget{GPIO_8c_ace8fd580466277115b8b7d913d050823}\label{GPIO_8c_ace8fd580466277115b8b7d913d050823}} 
\index{G\+P\+I\+O.\+c@{G\+P\+I\+O.\+c}!G\+P\+I\+O\+\_\+\+Global\+Interrupt\+Enable@{G\+P\+I\+O\+\_\+\+Global\+Interrupt\+Enable}}
\index{G\+P\+I\+O\+\_\+\+Global\+Interrupt\+Enable@{G\+P\+I\+O\+\_\+\+Global\+Interrupt\+Enable}!G\+P\+I\+O.\+c@{G\+P\+I\+O.\+c}}
\subsubsection{\texorpdfstring{G\+P\+I\+O\+\_\+\+Global\+Interrupt\+Enable()}{GPIO\_GlobalInterruptEnable()}}
{\footnotesize\ttfamily void G\+P\+I\+O\+\_\+\+Global\+Interrupt\+Enable (\begin{DoxyParamCaption}\item[{\hyperlink{structGPIO}{G\+P\+IO}$\ast$}]{device }\end{DoxyParamCaption})}



Abilitazione interrupt globali;. 


\begin{DoxyParams}{Parameters}
{\em device} & puntatore a struttura \hyperlink{structGPIO}{G\+P\+IO}, che si riferisce al device su cui operare \\
\hline
\end{DoxyParams}
\mbox{\Hypertarget{GPIO_8c_ad1d3c415a8f65c105fb7d4baf4e29e5c}\label{GPIO_8c_ad1d3c415a8f65c105fb7d4baf4e29e5c}} 
\index{G\+P\+I\+O.\+c@{G\+P\+I\+O.\+c}!G\+P\+I\+O\+\_\+\+Init@{G\+P\+I\+O\+\_\+\+Init}}
\index{G\+P\+I\+O\+\_\+\+Init@{G\+P\+I\+O\+\_\+\+Init}!G\+P\+I\+O.\+c@{G\+P\+I\+O.\+c}}
\subsubsection{\texorpdfstring{G\+P\+I\+O\+\_\+\+Init()}{GPIO\_Init()}}
{\footnotesize\ttfamily int G\+P\+I\+O\+\_\+\+Init (\begin{DoxyParamCaption}\item[{\hyperlink{structGPIO}{G\+P\+IO}$\ast$}]{G\+P\+I\+O\+\_\+device,  }\item[{struct module $\ast$}]{owner,  }\item[{struct platform\+\_\+device $\ast$}]{pdev,  }\item[{struct class$\ast$}]{class,  }\item[{const char$\ast$}]{driver\+\_\+name,  }\item[{const char$\ast$}]{device\+\_\+name,  }\item[{uint32\+\_\+t}]{serial,  }\item[{struct file\+\_\+operations $\ast$}]{f\+\_\+ops,  }\item[{irq\+\_\+handler\+\_\+t}]{irq\+\_\+handler,  }\item[{uint32\+\_\+t}]{irq\+\_\+mask }\end{DoxyParamCaption})}



Inizializza una struttura \hyperlink{structGPIO}{G\+P\+IO} per il corrispondente device. 


\begin{DoxyParams}{Parameters}
{\em G\+P\+I\+O\+\_\+device} & puntatore a struttura \hyperlink{structGPIO}{G\+P\+IO}, corrispondente al device su cui operare \\
\hline
{\em owner} & puntatore a struttura struct module, proprietario del device (T\+H\+I\+S\+\_\+\+M\+O\+D\+U\+LE) \\
\hline
{\em pdev} & puntatore a struct platform\+\_\+device \\
\hline
{\em driver\+\_\+name} & nome del driver \\
\hline
{\em device\+\_\+name} & nome del device \\
\hline
{\em serial} & numero seriale del device \\
\hline
{\em f\+\_\+ops} & puntatore a struttura struct file\+\_\+operations, specifica le funzioni che agiscono sul device \\
\hline
{\em irq\+\_\+handler} & puntatore irq\+\_\+handler\+\_\+t alla funzione che gestisce gli interrupt generati dal device \\
\hline
{\em irq\+\_\+mask} & maschera delle interruzioni attive del device\\
\hline
\end{DoxyParams}

\begin{DoxyRetVals}{Return values}
{\em 0} & se non si è verificato nessun errore \\
\hline
\end{DoxyRetVals}
Alloca un range di Mj e min numbers per il device a caratteri

Inizializza la struttura cdev specificando la struttura file operations associata al device a caratteri

Crea il device all\textquotesingle{}interno del filesystem assegnandogli i numbers richiesti in precedenza e ne restituisce il puntatore.

Aggiunge il device a caratteri al sistema. Se l\textquotesingle{}operazione va a buon fine sarà possibile vedere il device sotto /dev

Inizializza la struct resource con il valori recuperati dal device tree corrispondente al device

Alloca una quantita res\+\_\+size di memoria fisica per il dispositivo IO a partire dall\textquotesingle{}inidirzzo res.\+start e ne resituisce l\textquotesingle{}inidirizzo

Mappa la memoria fisca allocata e restituisce l\textquotesingle{}indirizzo virtuale

Cerca le specifiche dell\textquotesingle{}interrupt nel device tree e restituisce il suo numero identificativo

Inizializzazione della wait-\/queue per la system-\/call read() e poll()

Inizializzazione degli spinlock

Abilitazione degli interrupt del device \mbox{\Hypertarget{GPIO_8c_a0c89651b9f8711b02a3584f1968fac3b}\label{GPIO_8c_a0c89651b9f8711b02a3584f1968fac3b}} 
\index{G\+P\+I\+O.\+c@{G\+P\+I\+O.\+c}!G\+P\+I\+O\+\_\+\+Pending\+Pin\+Interrupt@{G\+P\+I\+O\+\_\+\+Pending\+Pin\+Interrupt}}
\index{G\+P\+I\+O\+\_\+\+Pending\+Pin\+Interrupt@{G\+P\+I\+O\+\_\+\+Pending\+Pin\+Interrupt}!G\+P\+I\+O.\+c@{G\+P\+I\+O.\+c}}
\subsubsection{\texorpdfstring{G\+P\+I\+O\+\_\+\+Pending\+Pin\+Interrupt()}{GPIO\_PendingPinInterrupt()}}
{\footnotesize\ttfamily unsigned G\+P\+I\+O\+\_\+\+Pending\+Pin\+Interrupt (\begin{DoxyParamCaption}\item[{\hyperlink{structGPIO}{G\+P\+IO}$\ast$}]{device }\end{DoxyParamCaption})}



Fornisce una maschera che indica quali interrupt non sono ancora stati serviti e che quindi risultano pending. 


\begin{DoxyParams}{Parameters}
{\em device} & puntatore a struttura \hyperlink{structGPIO}{G\+P\+IO}, che si riferisce al device su cui operare\\
\hline
\end{DoxyParams}
\begin{DoxyReturn}{Returns}
maschera riportante i pin per i quali gli interrupt non sono stati ancora serviti 
\end{DoxyReturn}
\mbox{\Hypertarget{GPIO_8c_af8147f170ccf8ee218d04b6aca631ba3}\label{GPIO_8c_af8147f170ccf8ee218d04b6aca631ba3}} 
\index{G\+P\+I\+O.\+c@{G\+P\+I\+O.\+c}!G\+P\+I\+O\+\_\+\+Pin\+Interrupt\+Ack@{G\+P\+I\+O\+\_\+\+Pin\+Interrupt\+Ack}}
\index{G\+P\+I\+O\+\_\+\+Pin\+Interrupt\+Ack@{G\+P\+I\+O\+\_\+\+Pin\+Interrupt\+Ack}!G\+P\+I\+O.\+c@{G\+P\+I\+O.\+c}}
\subsubsection{\texorpdfstring{G\+P\+I\+O\+\_\+\+Pin\+Interrupt\+Ack()}{GPIO\_PinInterruptAck()}}
{\footnotesize\ttfamily void G\+P\+I\+O\+\_\+\+Pin\+Interrupt\+Ack (\begin{DoxyParamCaption}\item[{\hyperlink{structGPIO}{G\+P\+IO}$\ast$}]{device,  }\item[{unsigned}]{mask }\end{DoxyParamCaption})}



Invia al device notifica di servizio di un interrupt;. 


\begin{DoxyParams}{Parameters}
{\em device} & puntatore a struttura \hyperlink{structGPIO}{G\+P\+IO}, che si riferisce al device su cui operare\\
\hline
{\em mask} & maschera di selezione degli interrupt da notificare \\
\hline
\end{DoxyParams}
\mbox{\Hypertarget{GPIO_8c_aa6651afb078323181b89b424bdae1b32}\label{GPIO_8c_aa6651afb078323181b89b424bdae1b32}} 
\index{G\+P\+I\+O.\+c@{G\+P\+I\+O.\+c}!G\+P\+I\+O\+\_\+\+Pin\+Interrupt\+Disable@{G\+P\+I\+O\+\_\+\+Pin\+Interrupt\+Disable}}
\index{G\+P\+I\+O\+\_\+\+Pin\+Interrupt\+Disable@{G\+P\+I\+O\+\_\+\+Pin\+Interrupt\+Disable}!G\+P\+I\+O.\+c@{G\+P\+I\+O.\+c}}
\subsubsection{\texorpdfstring{G\+P\+I\+O\+\_\+\+Pin\+Interrupt\+Disable()}{GPIO\_PinInterruptDisable()}}
{\footnotesize\ttfamily void G\+P\+I\+O\+\_\+\+Pin\+Interrupt\+Disable (\begin{DoxyParamCaption}\item[{\hyperlink{structGPIO}{G\+P\+IO}$\ast$}]{device,  }\item[{unsigned}]{mask }\end{DoxyParamCaption})}



Disabilitazione interrupt per i singoli pin del device. 


\begin{DoxyParams}{Parameters}
{\em device} & puntatore a struttura \hyperlink{structGPIO}{G\+P\+IO}, che si riferisce al device su cui operare\\
\hline
{\em mask} & maschera di selezione degli interrupt da disabilitare \\
\hline
\end{DoxyParams}
\mbox{\Hypertarget{GPIO_8c_a2fa8f2472358735588b580dea628a0c6}\label{GPIO_8c_a2fa8f2472358735588b580dea628a0c6}} 
\index{G\+P\+I\+O.\+c@{G\+P\+I\+O.\+c}!G\+P\+I\+O\+\_\+\+Pin\+Interrupt\+Enable@{G\+P\+I\+O\+\_\+\+Pin\+Interrupt\+Enable}}
\index{G\+P\+I\+O\+\_\+\+Pin\+Interrupt\+Enable@{G\+P\+I\+O\+\_\+\+Pin\+Interrupt\+Enable}!G\+P\+I\+O.\+c@{G\+P\+I\+O.\+c}}
\subsubsection{\texorpdfstring{G\+P\+I\+O\+\_\+\+Pin\+Interrupt\+Enable()}{GPIO\_PinInterruptEnable()}}
{\footnotesize\ttfamily void G\+P\+I\+O\+\_\+\+Pin\+Interrupt\+Enable (\begin{DoxyParamCaption}\item[{\hyperlink{structGPIO}{G\+P\+IO}$\ast$}]{device,  }\item[{unsigned}]{mask }\end{DoxyParamCaption})}



Abilitazione interrupt per i singoli pin del device. 


\begin{DoxyParams}{Parameters}
{\em device} & puntatore a struttura \hyperlink{structGPIO}{G\+P\+IO}, che si riferisce al device su cui operare \\
\hline
{\em mask} & maschera di selezione degli interrupt da abilitare \\
\hline
\end{DoxyParams}
\mbox{\Hypertarget{GPIO_8c_a1bee66d415014122270d76e4c2192842}\label{GPIO_8c_a1bee66d415014122270d76e4c2192842}} 
\index{G\+P\+I\+O.\+c@{G\+P\+I\+O.\+c}!G\+P\+I\+O\+\_\+\+Reset\+Can\+Read@{G\+P\+I\+O\+\_\+\+Reset\+Can\+Read}}
\index{G\+P\+I\+O\+\_\+\+Reset\+Can\+Read@{G\+P\+I\+O\+\_\+\+Reset\+Can\+Read}!G\+P\+I\+O.\+c@{G\+P\+I\+O.\+c}}
\subsubsection{\texorpdfstring{G\+P\+I\+O\+\_\+\+Reset\+Can\+Read()}{GPIO\_ResetCanRead()}}
{\footnotesize\ttfamily void G\+P\+I\+O\+\_\+\+Reset\+Can\+Read (\begin{DoxyParamCaption}\item[{\hyperlink{structGPIO}{G\+P\+IO}$\ast$}]{device }\end{DoxyParamCaption})}



Utilizzata per resettare il flag \char`\"{}can\+\_\+read\char`\"{} di uno specifico device \hyperlink{structGPIO}{G\+P\+IO}. 


\begin{DoxyParams}{Parameters}
{\em device} & puntatore a struttura \hyperlink{structGPIO}{G\+P\+IO}, che si riferisce al device su cui operare \\
\hline
\end{DoxyParams}
\mbox{\Hypertarget{GPIO_8c_af9854f82763d67da8ebb06b7aa10c064}\label{GPIO_8c_af9854f82763d67da8ebb06b7aa10c064}} 
\index{G\+P\+I\+O.\+c@{G\+P\+I\+O.\+c}!G\+P\+I\+O\+\_\+\+Set\+Can\+Read@{G\+P\+I\+O\+\_\+\+Set\+Can\+Read}}
\index{G\+P\+I\+O\+\_\+\+Set\+Can\+Read@{G\+P\+I\+O\+\_\+\+Set\+Can\+Read}!G\+P\+I\+O.\+c@{G\+P\+I\+O.\+c}}
\subsubsection{\texorpdfstring{G\+P\+I\+O\+\_\+\+Set\+Can\+Read()}{GPIO\_SetCanRead()}}
{\footnotesize\ttfamily void G\+P\+I\+O\+\_\+\+Set\+Can\+Read (\begin{DoxyParamCaption}\item[{\hyperlink{structGPIO}{G\+P\+IO}$\ast$}]{device }\end{DoxyParamCaption})}



Utilizzata per asserire il flag \char`\"{}can\+\_\+read\char`\"{} di uno specifico device \hyperlink{structGPIO}{G\+P\+IO}. 


\begin{DoxyParams}{Parameters}
{\em device} & puntatore a struttura \hyperlink{structGPIO}{G\+P\+IO}, device su cui operare \\
\hline
\end{DoxyParams}
\mbox{\Hypertarget{GPIO_8c_abe7a165f864923f292b33d2d25fa8525}\label{GPIO_8c_abe7a165f864923f292b33d2d25fa8525}} 
\index{G\+P\+I\+O.\+c@{G\+P\+I\+O.\+c}!G\+P\+I\+O\+\_\+\+Test\+Can\+Read\+And\+Sleep@{G\+P\+I\+O\+\_\+\+Test\+Can\+Read\+And\+Sleep}}
\index{G\+P\+I\+O\+\_\+\+Test\+Can\+Read\+And\+Sleep@{G\+P\+I\+O\+\_\+\+Test\+Can\+Read\+And\+Sleep}!G\+P\+I\+O.\+c@{G\+P\+I\+O.\+c}}
\subsubsection{\texorpdfstring{G\+P\+I\+O\+\_\+\+Test\+Can\+Read\+And\+Sleep()}{GPIO\_TestCanReadAndSleep()}}
{\footnotesize\ttfamily void G\+P\+I\+O\+\_\+\+Test\+Can\+Read\+And\+Sleep (\begin{DoxyParamCaption}\item[{\hyperlink{structGPIO}{G\+P\+IO}$\ast$}]{device }\end{DoxyParamCaption})}



Testa il valore del flag \char`\"{}can\+\_\+read\char`\"{}. Se è uguale a 0, ovvero non è possibile effettuare una lettura, mette in sleep il processo. 


\begin{DoxyParams}{Parameters}
{\em device} & puntatore a struttura \hyperlink{structGPIO}{G\+P\+IO}, che si riferisce al device su cui operare \\
\hline
\end{DoxyParams}
\mbox{\Hypertarget{GPIO_8c_a2055d1dae8287604bd216a00b45fbb80}\label{GPIO_8c_a2055d1dae8287604bd216a00b45fbb80}} 
\index{G\+P\+I\+O.\+c@{G\+P\+I\+O.\+c}!G\+P\+I\+O\+\_\+\+Wake\+Up@{G\+P\+I\+O\+\_\+\+Wake\+Up}}
\index{G\+P\+I\+O\+\_\+\+Wake\+Up@{G\+P\+I\+O\+\_\+\+Wake\+Up}!G\+P\+I\+O.\+c@{G\+P\+I\+O.\+c}}
\subsubsection{\texorpdfstring{G\+P\+I\+O\+\_\+\+Wake\+Up()}{GPIO\_WakeUp()}}
{\footnotesize\ttfamily void G\+P\+I\+O\+\_\+\+Wake\+Up (\begin{DoxyParamCaption}\item[{\hyperlink{structGPIO}{G\+P\+IO}$\ast$}]{device }\end{DoxyParamCaption})}



Risveglia i processi in attesa sulle code di read e poll. 


\begin{DoxyParams}{Parameters}
{\em device} & puntatore a struttura \hyperlink{structGPIO}{G\+P\+IO}, che si riferisce al device su cui operare \\
\hline
\end{DoxyParams}
